\documentclass[12pt]{article}
\usepackage[utf8]{inputenc}
\usepackage[spanish]{babel}
\usepackage{amsmath, amssymb}
\usepackage{geometry}
\usepackage{hyperref}
\usepackage{graphicx}
\usepackage{enumitem}
\geometry{margin=2.5cm}

\title{Recomendaciones y Pruebas Eléctricas para Sensores Alphasense}
\author{}
\date{\today}

\begin{document}

\maketitle

\section*{1. Objetivo}
Establecer una metodología de prueba para verificar condiciones eléctricas, consumo y niveles de ruido en sensores de gases electroquímicos Alphasense. Se busca identificar la calidad de alimentación y tierra, así como caracterizar el comportamiento eléctrico de cada sensor de forma individual y repetir pruebas para todos los sensores conectados en serie, revisar .

\section*{2. Equipamiento Recomendado}
\begin{itemize}[label=\textbullet]
  \item Multímetro de precisión (resolución de al menos 1 mV / 1 µA)
  \item Osciloscopio (≥20 MHz de ancho de banda, con opción AC/DC y modo diferencial)
  \item Sonda de corriente (ej. tipo hall o shunt de bajo valor con amplificador)
  \item Fuente regulada de laboratorio con bajo ruido(usar Nordic PPK para pruebas individuales)
  \item Cables blindados y clips Kelvin (para medidas diferenciales)
\end{itemize}

\section*{3. Mediciones a Realizar}

\subsection*{3.1. Medición de Voltaje de Alimentación (V\textsubscript{IN})}
\begin{itemize}
  \item Medir tensión DC entre pin VCC y GND del conector ISB.
  \item Repetir la medición en la entrada y salida del regulador LDO.
  \item Registrar valores típicos y desviaciones bajo carga.
  \item Registrar valores de ruido de gnd pines de gnd.
\end{itemize}

\subsection*{3.2. Medición de Corriente de Consumo}
\begin{itemize}
  \item Usar Multimetro en serie.
  \item Alternativamente, utilizar sonda de corriente si se desea no invadir la línea.
\end{itemize}

\subsection*{3.3. Medición de Ruido en la Alimentación}
\begin{itemize}
  \item Conectar el osciloscopio en modo AC entre VCC y GND del sensor.
  \item Evaluar rizado de alta frecuencia (>10 kHz), spikes o modulación.
  \item Repetir en la salida del LDO y entrada del sensor.
\end{itemize}

\subsection*{3.4. Medición de Ruido en la Línea de Tierra (GND)}
\begin{itemize}
  \item Configurar osciloscopio en modo diferencial(revisar manual del osiliscopio).
  \item Medir entre GND del sensor y GND central del sistema (fuente o plano principal).
  \item Identificar diferencias de potencial o corrientes parásitas.
\end{itemize}

\section*{4. Consideraciones}
\begin{itemize}
  \item Asegurar conexiones de baja impedancia y minimizar bucles de masa.
  \item Utilizar cables cortos y bien acoplados.
  \item Evitar puntos de masa flotantes entre sensor y regulador.
  \item Repetir pruebas con carga activa (sensor funcionando) y en reposo.
\end{itemize}

\section*{5. Registro de Datos}
Crear una tabla por sensor con los siguientes campos:
\begin{itemize}
  \item ID del sensor
  \item Tensión entrada (antes del LDO)
  \item Tensión salida (en el sensor)
  \item Corriente media
  \item Rizado pico a pico observado
  \item Diferencia de potencial GND–GND (sensor–fuente)
\begin{figure}
    \centering
    \includegraphics[width=0.5\linewidth]{image.png}
    \caption{Revisar continuadad de GND y RUIDO en pin 2, 4 y 6}
    \label{fig:placeholder}
\end{figure}
\end{itemize}

\end{document}
