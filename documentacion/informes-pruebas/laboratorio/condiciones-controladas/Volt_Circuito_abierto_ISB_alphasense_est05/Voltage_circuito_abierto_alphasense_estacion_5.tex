%-------------------------------------------------------
%   REPORTE TÉCNICO – FORMATO MODERNO (PDFLATEX)
%-------------------------------------------------------
\documentclass[11pt]{article}

%-------------------------------------------------------
%   PAQUETES DE DISEÑO COMPATIBLES
%-------------------------------------------------------
\usepackage[a4paper,margin=2.2cm]{geometry}
\usepackage[T1]{fontenc}
\usepackage[utf8]{inputenc}
\usepackage{lmodern} % Fuente moderna compatible
\usepackage{titlesec}
\usepackage{microtype}
\usepackage{setspace}
\usepackage{paracol}
\usepackage{graphicx}
\usepackage{booktabs}
\usepackage{tcolorbox}
\usepackage{hyperref}
\usepackage{multicol}
\usepackage{siunitx}
\usepackage{xcolor}
\usepackage[siunitx]{circuitikz}

\usepackage{subcaption}
\usepackage{caption}

%-------------------------------------------------------
%   COLORES PERSONALIZADOS
%-------------------------------------------------------
\definecolor{uddblue}{HTML}{003366}
\definecolor{graytext}{HTML}{3A3A3A}

%-------------------------------------------------------
%   ESTILO DE TÍTULOS
%-------------------------------------------------------
\titleformat{\section}{\Large\bfseries\color{uddblue}}{}{0em}{}[\titlerule]
\titleformat{\subsection}{\large\bfseries\color{uddblue}}{}{0em}{}
\titleformat{\subsubsection}{\normalsize\bfseries\color{uddblue}}{}{0em}{}

% Hipervínculos
\hypersetup{
	colorlinks=true,
	linkcolor=uddblue,
	urlcolor=uddblue,
}

% Cuadros modernos
\tcbset{
	sharp corners,
	colback=white,
	colframe=uddblue,
	left=8pt,
	right=8pt,
	top=6pt,
	bottom=6pt,
}

	
%-------------------------------------------------------
%   DOCUMENTO
%-------------------------------------------------------
\begin{document}
	
	%-------------------------------------------------------
	% PORTADA
	%-------------------------------------------------------
	\begin{center}
		{\huge \bfseries Medición de Offset Electrónico}\\[4pt]
		{\Large \bfseries (Open Circuit Voltage – OCV)}\\[8pt]
		\rule{0.9\textwidth}{1pt}\\[10pt]
	\end{center}
\begin{figure}[h]
	\centering
	\includegraphics[width=0.7\linewidth]{img/IMG_1555}
	\caption{Placa Alphasense ISB}
	\label{fig:img1555}
\end{figure}


	
	\begin{tcolorbox}
		\textbf{Responsable:} Luis Gómez (\href{mailto:luis.gomez@udd.cl}{luis.gomez@udd.cl})\\
		\textbf{Revisores:} Alejandro Rebolledo (\href{mailto:arebolledo@udd.cl}{arebolledo@udd.cl})\\ % me agrego como revisor
		\textbf{Lugar:} Laboratorio de Electrónica C+ — UDD, Santiago\\
		\textbf{Fecha:} 11 noviembre 2025 (10:00–13:52)\\
		\textbf{Estación evaluada:} Estación 5\\
		\textbf{Proyecto:} Linea base pública para la región metropolitana
	\end{tcolorbox}
	
	%-------------------------------------------------------
	\section{Propósito del Test}
	
	El objetivo del ensayo fue cuantificar el \textbf{offset electrónico intrínseco} generado por la placa
	Alphasense ISB cuando opera \textbf{sin sensor conectado} (condición de circuito abierto).  
	Este voltaje corresponde exclusivamente a la acción de la electrónica interna del módulo 
	(amplificadores, referencias y filtros) y constituye el valor base que debe restarse a las mediciones reales del sensor para separar:
	\begin{itemize}
		\item el offset electrónico (\textbf{WEe}, \textbf{AEe}),
		\item el offset propio del sensor (\textbf{WEo}, \textbf{AEo}).
	\end{itemize}
	
	Según la documentación técnica de Alphasense, el rango típico esperado para los canales OP1 (WE)
	y OP2 (Aux) se encuentra entre \textbf{200 y 300 mV} (ver figura).
	
	
	
\begin{figure}
	\centering
	\includegraphics[width=0.4\linewidth]{"img/Conexiones_ISB.png"}
	\caption{Conexiones ISB Alphasense ISB }
	\label{fig:captura-de-pantalla-de-2025-11-12-20-29-07}
\end{figure}
	%-------------------------------------------------------
	\pagebreak
	\section{Procedimiento}
	
	\subsection*{Condiciones del ensayo (sin sensor B4)}
	\begin{itemize}
		\item Alimentación aplicada: 3.5–6.5 VDC (fuente de bajo ruido).
		\item Medición de las señales OP1 (WE) y OP2 (Aux).
		\item Evaluación exclusivamente del offset electrónico en estado de circuito abierto.
	\end{itemize}
	
	\subsection*{Equipamiento}
	\begin{itemize}
		\item Power Profile Kit II (fuente de alimentación de bajo ruido).
		\item Osciloscopio GW Instek GDS-12028.
		\item Multímetro Keysight 34461A.
	\end{itemize}

\begin{figure}[h!]
	\centering
	
	\begin{subfigure}[b]{0.32\textwidth}
		\centering
		\includegraphics[width=\textwidth]{img/fuente.jpg}
		\caption{Fuente de alimentación}
		\label{fig:fuente}
	\end{subfigure}
	\hfill
	\begin{subfigure}[b]{0.32\textwidth}
		\centering
		\includegraphics[width=\textwidth]{img/tester.jpg}
		\caption{Multímetro Keysight}
		\label{fig:tester}
	\end{subfigure}
	\hfill
	\begin{subfigure}[b]{0.32\textwidth}
		\centering
		\includegraphics[width=\textwidth]{img/osciloscopio.jpg}
		\caption{Osciloscopio GW Instek}
		\label{fig:osciloscopio}
	\end{subfigure}
	
	\caption{Equipamiento utilizado en el ensayo: fuente de bajo ruido, multímetro y osciloscopio.}
	\label{fig:equipamiento}
\end{figure}


	
	\subsection*{Procedimiento medición aplicado a cada placa ISB}
	\begin{itemize}
		\item Se extrae el sensor del socket de la placa Alphasense ISB. Esto podria no ser recomendable ya que se podría perder la calibración de fabrica.
		\item Se conectan los cables según diagrama de la figura \ref{fig:conexiones} y \ref{fig:conexiones2}.%agregar figura de conexiones y agregar esquematicos
		\item Se conecta el terminal GND de la placa ISB al terminal negativo de la fuente de bajo ruido.
		\item Se conectan las tierras de los canales 1 y 2 del osciloscopio al GND de la placa.
		\item Se conectan OP1+ y OP2+ a las puntas de prueba de los canales 1 y 2 del osciloscopio, respectivamente.
		\item Se enciende la fuente de alimentación.
		\item Tras un tiempo de estabilización de 1 segundo, se conecta el terminal positivo de la fuente al pin V+ de la ISB.
		\item Se realizan las mediciones con osciloscopio y multímetro.
		\item Se apaga la fuente de alimentación y se desconecta la placa.
	\end{itemize}
	
	Este procedimiento se repitió para cinco placas ISB diferentes, identificadas como DBI 01 a DBI 05 mediante etiquetas adhesivas.
	
	
\begin{figure}[h]
	\centering
	\includegraphics[width=0.7\linewidth]{img/conexiones}
	\caption{Conexiones de la prueba de Offset}
	\label{fig:conexiones}
\end{figure}

%-------------------------------
\begin{figure}[h!]
	\centering
	\begin{circuitikz}[scale=1.1]
		% ============================
		% BUS DE TIERRA (GND) COMÚN
		% ============================
		\draw (-1,-0.5) node[ground]{} to[short] (7.3,-0.5);
		\node[below] at (5,-0.5) {GND común};
		
		% ============================
		% FUENTE DC A LA IZQUIERDA
		% ============================
		% Batería entre GND y V+
		\draw (-1,-0.5) to[battery1,l=$V_\text{fuente}$] (-1,2);
		% V+ hacia la derecha
		\draw (-1,2) -- (1.5,2);
		\node[above] at (0.2,2) {V+};
		\node[below] at (0.2,-0.5) {GND fuente};
		
		% ============================
		% ISB AL CENTRO
		% ============================
		% Caja ISB más grande
		\draw (2.5,0) rectangle (6,3.5);
		\node at (4.25,3.7) {\textbf{ISB (Molex)}};
		
		% Pin V+ ISB
		\draw (2.5,3) -- (1.5,3);
		\node[left] at (4.1,3) {V+ IN};
		% Conexión V+ fuente → V+ ISB
		\draw (1.5,2) -- (1.5,3);
		
		% Pin GND ISB
		\draw (2.5,0.8) -- (1.8,0.8);
		\node[left] at (4.3,0.8) {GND ISB};
		\draw (1.8,0.8) -- (1.8,-0.5);
		
		% Pines de salida con mejor espaciado
		\draw (6,3.2) -- (7.2,3.2);
		\node[right] at (4.8,3.2) {OP1+};
		\draw (6,1.8) -- (7,1.8);
		\node[right] at (4.8,1.8) {OP2+};
		
		% ============================
		% OSCILOSCOPIO A LA DERECHA
		% ============================
		% Caja osciloscopio
		\draw (8,1.) rectangle (10,4.);
		\node at (9.2,4.2) {\textbf{Osciloscopio}};
		
		% Terminales del osciloscopio
		% Canal 1 (+)
		\draw (8,3.7) -- (7.3,3.7);
		\node[left] at (9.5,3.7) {Ch1 (+)};
		% Canal 1 (–)
		\draw (8,3.0) -- (7.3,3.0);
		\node[left] at (9.5,3.0) {Ch2 (+)};
		% GND del osciloscopio
		\draw (8,1.5) -- (7.3,1.5);
		\node[left] at (9.7,1.5) {GND osc.};
		
		% ============================
		% CONEXIONES SEÑALES → OSCILO (SIN CRUCES)
		% ============================
		% OP1+ → Ch1 (+) - línea directa con segmentos limpios
		\draw (7.2,3.2) -- (7.25,3.2) -- (7.25,3.7) -- (7.3,3.7);
		
		% OP2+ → Ch1 (–) - línea con curva para evitar cruces
		\draw  (7,1.8) -- (7.3,1.8) -- (7.3,3.0);
		
		% GND común → GND osciloscopio
		\draw (7.3,1.5) -- (7.3,-0.5);
		
	\end{circuitikz}
	\caption{Esquema de conexiones electrónicas para la medición de offset en la ISB.}
	\label{fig:conexiones2}
\end{figure}




	
	%-------------------------------------------------------
	\newpage
	\section{Resultados}
	
	

	
	\subsection{Mediciones con multímetro (TESTER)}
	
	Resultados obtenidos mediante mediciones con osciloscopio (ver tabla \ref{tab:tester}) 
	
\begin{table}[h!]
	\centering
	\caption{\textcolor{uddblue}{Valores medidos con multímetro (OP1 y OP2 en condición de circuito abierto).}}
	\label{tab:tester}
	\vspace{4pt}
	\begin{tabular}{@{}lccc@{}}
		\toprule
		\textbf{DBI} & \textbf{Gas} & \textbf{OP1+ (mV)} & \textbf{OP2+ (mV)} \\
		\midrule
		01 & CO     & 352 & 341 \\
		02 & NO$_2$ & 236 & 237 \\
		03 & NO     & 273 & 277 \\
		04 & OX     & 231 & 242 \\
		05 & SO$_2$ & 354 & 359 \\
		\bottomrule
	\end{tabular}
\end{table}



	
\subsection{Mediciones con osciloscopio}

Resultados obtenidos mediante mediciones con osciloscopio (ver tabla \ref{tab:osciloscopio} y´ figura \ref{fig:medicionosciloscopio}) 
	
\begin{table}[h]
	\centering
	\caption{Valores de OP1+ y OP2+ medidos con osciloscopio durante la prueba de offset electrónico.}
	\label{tab:osciloscopio}
	\begin{tabular}{lccc}
		\toprule
		\textbf{DBI} & \textbf{Gas} & \textbf{OP1+ (mV)} & \textbf{OP2+ (mV)} \\
		\midrule
		01 & CO     & 375 & 370 \\
		02 & NO$_2$ & 267 & 272 \\
		03 & NO     & 312 & 320 \\
		04 & OX     & 257 & 270 \\
		05 & SO$_2$ & 384 & 397 \\
		\bottomrule
	\end{tabular}
\end{table}
	
	\begin{figure}[h]
		\centering
		\includegraphics[width=0.7\linewidth]{img/Medicion_Osciloscopio.jpeg}
		\caption{salida osciloscopio de una de las pruebas para OP1+ y OP2+}
		\label{fig:medicionosciloscopio}
	\end{figure}
	
	
	%-------------------------------------------------------
\newpage
	\section{Análisis}
	
	Las mediciones obtenidas con multímetro presentan una señal estable y con menor efecto del
	ruido eléctrico, por lo que representan una mejor estimación del offset electrónico en 
	estado estacionario.
	
	Los valores correspondientes a NO$_2$, NO y OX se encuentran dentro del rango esperado 
	(200–300 mV). En contraste, las placas asociadas a CO y SO$_2$ presentan offsets superiores, 
	aunque los valores son estables y coherentes entre OP1 y OP2, indicando un comportamiento 
	normal de la electrónica del módulo.
	
	Las mediciones con osciloscopio son ligeramente mayores debido a:
	\begin{itemize}
		\item sensibilidad al ripple de la alimentación,
		\item captura de transientes breves,
		\item menor impedancia de entrada en comparación al multímetro.
	\end{itemize}
	
	%-------------------------------------------------------
	\section{Conclusiones}
	
	\begin{itemize}
		\item Las mediciones realizadas con multímetro se recomiendan para establecer los valores finales de offset electrónico \textbf{WEe} y \textbf{AEe}.
		\item No se observaron signos de saturación, fallas en amplificadores ni comportamientos anómalos en las placas ISB.
		\item Los valores caracterizados pueden ser utilizados directamente como parte del proceso de corrección electrónica previo al cálculo de concentración.
	\end{itemize}
	
	%-------------------------------------------------------
	\section{Recomendaciones}
	
	\begin{tcolorbox}
		\textbf{Para mejorar futuros ensayos, se recomienda:}
		\begin{itemize}
			\item añadir capacitores de desacople adicionales en la salida de la ISB para reducir el ruido,
			\item realizar un muestreo temporal prolongado y calcular la desviación estándar del offset,
			\item blindar cables de conexión,
			\item controlar la temperatura de la sala, ya que el offset de la electrónica presenta dependencia térmica.
			\item Medir con sensor conectado y sin gas para comparar el offset electrónico con el offset total (electrónico + sensor).%argrego recomendaciones
			\item Realizar mediciones de corriente, con y sin sensor, para evaluar el consumo en estado de circuito abierto. %argrego recomendaciones
			\item Armar pares de placa y sensor y no separarlos.
		\end{itemize}
	\end{tcolorbox}

-------------------------------------------------------
%   Bibliografía
%-------------------------------------------------------

\begin{thebibliography}{99}
	
{\small 	\bibitem{AlphasenseISB}
	Alphasense Ltd.
	\textit{4-Electrode Individual Sensor Board (ISB) User Manual}.
	Issue 7, Document 085-2217, Great Notley, Essex, UK, 2019.
	
	\bibitem{AlphasenseAAN803}
	Alphasense Ltd.
	\textit{AAN 803-05: Correcting for Background Currents in Four-Electrode Toxic Gas Sensors}.
	Application Note, 2019.
	
	\bibitem{AlphasenseB4Series}
	Alphasense Ltd.
	\textit{B4-Series Toxic Gas Sensors: Specifications}.
	Sensor Technology House, Great Notley, Essex, UK, 2019.
	
	\bibitem{Keysight34461A}
	Keysight Technologies.
	\textit{Keysight 34461A Truevolt 6½ Digit Multimeter – User’s Guide}.
	2020.
	
	\bibitem{GWInstekGDS12000}
	GW Instek.
	\textit{GDS-12000 Series Digital Oscilloscopes – User Manual}.
	2021.
	
	\bibitem{NordicPPK2}
	Nordic Semiconductor.
	\textit{Power Profiler Kit II – User Guide}.
	2023.}
	
\end{thebibliography}
	
\end{document}

	


